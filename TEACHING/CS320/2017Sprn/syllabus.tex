\documentclass[11pt]{article}

\usepackage{latexsym}
%\usepackage{psfig}

\oddsidemargin=-0.75cm
\evensidemargin=-0.75cm
\textwidth=17.60cm
\textheight=22cm
\topmargin=-1cm
\headsep=1cm

\newtheorem{lemma}{Lemma}
\newtheorem{exercise}[lemma]{Exercise}

\begin{document}
%\hspace{-1cm} 
%\mbox{\psfig{figure=../../../BUseal.eps,height=1.125in,angle=0}}
%\vspace{-2.5cm}
\begin{flushright}
\begin{minipage}{430pt}
{
\begin{flushleft}
{~~\large\sc BU CAS CS 320 (Spring, 2017)} \\
{~~\LARGE\sc Concepts of Programming Languages}
\end{flushleft}
}
\end{minipage}
\end{flushright}

\vspace{12pt}
\begin{center}
\begin{minipage}{16cm}
\begin{center}
{\LARGE\bf Syllabus} \\[12pt]
\end{center}
\end{minipage}
\end{center}

\thispagestyle{empty}

\begin{itemize}
\item {\bf Semester} Spring, 2017

\item {\bf Lecture Times}: TR: 12:30-1:45pm

\item {\bf Classroom}: CAS 325

\item {\bf Instructor}: Hongwei Xi
\item {\bf Instructor's Office}: MCS 287
\item {\bf Instructor's Office Hours}: TBA

\item {\bf Teaching Fellow}: Hanwen Wu
\item {\bf Teaching Fellow's Office Hours}: TBA

\item {\bf Class Homepage}:\\
\centerline{%
\texttt{http://www.cs.bu.edu/\char126hwxi/academic/courses/CS320/Spring17/index.html}%
}%\centerline

\item {\bf Textbook}:\kern6pt
\begin{itemize}
\item
{\em Introduction to Programming in ATS} by Hongwei Xi, which is available at:
\begin{center}
\texttt{http://ats-lang.sourceforge.net/DOCUMENT/INT2PROGINATS/HTML/HTMLTOC/book1.html}
\end{center}
\end{itemize}

\item
{\bf Midterm 1}\kern6pt:
A take-home exam on Thursday, the 2nd of March, 2017.

\item
{\bf Midterm 2}\kern6pt:
A take-home exam on a date around April 20, 2017.

\item {\bf Final}:\kern6pt A two-hour in-class exam

\item {\bf Overview}:
Concepts of Programming Languages (CPL) is a course that introduces
students to some fundamental concepts in programming language design and
implementation. The primary goal is to allow students who complete this
subject to have a good feel for the elements of style and aesthetics of
programming and a good command of the major techniques for controlling
complexity in programming.

ATS is a programming language that makes pervasive use of types in
capturing programming invariants. We will be learning how datatypes in ATS
can be used to conveniently model data structures and how pattern matching
can be used to facilitate programming with datatypes. Also, we are to make
extensive use of abstract types in the construction of (relatively) large
and complex programs.

Ultimately, we would like to make a convincing argument that programming
can be a great deal of fun if you do it the right way!

\item {\bf Grades}
The final score is calculated using the following formula.
\[\begin{array}{rcl}
\mbox{final score} & = & \mbox{30\%$\cdot$(homework)} \\
                   & + & \mbox{20\%$\cdot$(midterm1)} \\
                   & + & \mbox{20\%$\cdot$(midterm2)} \\
                   & + & \mbox{20\%$\cdot$(final)} \\
                   & + & \mbox{10\%$\cdot$(class participation)} \\
\end{array}\]
The final letter grade is calculated as follows.
\begin{itemize}
\item{\bf A}: final score is $80\%$ or above (A-, A)
\item{\bf B}: final score is $70\%$ or above (B-, B, B+)
\item{\bf C}: final score is $60\%$ or above (C-, C, C+)
\item{\bf D}: final score is $50\%$ or above (D)
\end{itemize}

\item{\bf Homework Assignments}
There will be a homework assignment every one or two weeks depending on the
amount of effort and time needed to finish the assignment. An assignment that
is turned in after its due time is accepted but penalized according to the
following policy.
\begin{itemize}
\item 20\% point deduction if the assignment is turned in within 24 hours
after its due time.
\item 50\% point deduction if the assignment is turned in between 24 and 48 hours
after its due time.
\item no credit if the assignment is turned in more than 48 hours later after its due time.
\end{itemize}

\item{\bf Academic Integrity}:
We adhere strictly to the standard BU guidelines for academic
integrity. For this course, it is perfectly acceptable for you to discuss
the general concepts and principles behind an assignment with other
students. However, it is not proper, without prior authorization of the
instructor, to arrive at collective solutions. In such a case, each student
is expected to develop, write up and hand in an individual solution and, in
doing so, gain a sufficient understanding of the problem so as to be able
to explain it adequately to the instructor.  Under {\em no} circumstances
should a student copy, partly or wholly, the completed solution of another
student. If one makes substantial use of certain code that is not written by
oneself, then the person must explicitly mention the source of the involved
code.

\end{itemize}

\end{document}

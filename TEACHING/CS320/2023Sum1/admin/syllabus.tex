\documentclass[11pt]{article}

\usepackage{latexsym}
%\usepackage{psfig}

\oddsidemargin=-0.75cm
\evensidemargin=-0.75cm
\textwidth=17.60cm
\textheight=22cm
\topmargin=-1cm
\headsep=1cm

\newtheorem{lemma}{Lemma}
\newtheorem{exercise}[lemma]{Exercise}

\begin{document}
%\hspace{-1cm} 
%\mbox{\psfig{figure=../../../BUseal.eps,height=1.125in,angle=0}}
%\vspace{-2.5cm}
\begin{flushright}
\begin{minipage}{430pt}
{
\begin{flushleft}
{~~\large\sc BU CAS CS 320 (Summer I, 2023)} \\
{~~\LARGE\sc Concepts of Programming Languages}
\end{flushleft}
}
\end{minipage}
\end{flushright}

\vspace{12pt}
\begin{center}
\begin{minipage}{16cm}
\begin{center}
{\LARGE\bf Syllabus} \\[12pt]
\end{center}
\end{minipage}
\end{center}

\thispagestyle{empty}

\begin{itemize}
\item {\bf Semester} Summer I, 2023

\item {\bf Lecture Times}: MTW: 1:00-3:30pm

\item {\bf Classroom}: SOC B63 $\rightarrow$ KCB 102

\item {\bf Instructor}: Hongwei Xi
\item {\bf Instructor's Office}: CDS 727
\item {\bf Instructor's Office Hours}: TBA

\item {\bf Teaching Fellow}: Qiancheng Fu
\item {\bf Teaching Fellows' Office Hours}: TBA

\item {\bf Teaching Assistants}: TBA
\item {\bf Teaching Assistants' Office Hours}: TBA

\item {\bf Textbook}:\kern6pt
\begin{itemize}
\item
{\em Programming in SML} by Robert Harper, which is available at
\begin{center}
\texttt{http://www.cs.cmu.edu/~rwh/isml/book.pdf}
\end{center}
\item
{\em Introduction to Computation and Programming Using Python} (3rd edition) by John V. Guttag
\end{itemize}

\item
{\bf Midterm 1}\kern6pt: 06/07/2023
\item
{\bf Midterm 2}\kern6pt: 06/23/2023

\item {\bf Final}:\kern6pt A two-hour in-class exam on 06/28/2023

\item {\bf Overview}:
Concepts of Programming Languages (CPL) is a course that introduces
students to some fundamental concepts in programming language design and
implementation. The primary goal is to allow students who complete this
subject to have a good feel for the elements of style and aesthetics of
programming and a good command of the major techniques for controlling
complexity in programming.

SML is a functional programming language that makes pervasive use of
types in capturing programming invariants. We will be starting with
SML and then making a gradual transition from SML to Python so as to
demonstrate concretely that the concepts learned in the context of SML
can be readily applied in the context of Python, one of the most
popular language in the world.

Ultimately, we would like to make a convincing argument that programming
can be a great deal of fun if you do it the right way!

\item {\bf Grades}
The final score is calculated using the following formula:
\[\begin{array}{rcl}
\mbox{final score} & = & \mbox{20\%$\cdot$(homework)} \\
                   & + & \mbox{15\%$\cdot$(quizzes)} \\
                   & + & \mbox{15\%$\cdot$(midterm1)} \\
                   & + & \mbox{15\%$\cdot$(midterm2)} \\
                   & + & \mbox{30\%$\cdot$(final)} \\
                   & + & \mbox{05\%$\cdot$(class participation)} \\
\end{array}\]
The final letter grade is calculated as follows.
\begin{itemize}
\item{\bf A}: final score is $85\%$ or above (A-, A)
\item{\bf B}: final score is $75\%$ or above (B-, B, B+)
\item{\bf C}: final score is $60\%$ or above (C-, C, C+)
\item{\bf D}: final score is $50\%$ or above (D)
\end{itemize}

\item{\bf Homework Assignments}
There will be a homework assignment every one or two weeks depending on the
amount of effort and time needed to finish the assignment. An assignment that
is turned in after its due time is accepted but penalized according to the
following policy.
\begin{itemize}
\item 20\% point deduction if the assignment is turned in within 24 hours
after its due time.
\item 50\% point deduction if the assignment is turned in between 24 and 48 hours
after its due time.
\item no credit if the assignment is turned in more than 48 hours later after its due time.
\end{itemize}

\item{\bf Explaining Your Solutions}:
From time to time, students may be requested to explain in details
their solutions to the instructor, the TFs, and/or the TAs. Those who
cannot adequately explain their solutions may see that their acquired
points be deducted partly or wholely.

\item{\bf Academic Integrity}:
We adhere strictly to the standard BU guidelines for academic
integrity. For this course, it is perfectly acceptable for you to
discuss the general concepts and principles behind an assignment with
other people and/or chatbots. However, it is not proper, without prior
authorization of the instructor, to arrive at collective solutions. In
such a case, each student is expected to develop, write up and hand in
an individual solution and, in doing so, gain a sufficient
understanding of the problem so as to be able to explain it adequately
to the instructor.  Under {\em no} circumstances should a student
copy, partly or wholly, the completed solution of another student. If
one makes substantial use of certain code that is not written by
oneself, then the person must explicitly mention the source of the
involved code.

\end{itemize}

\end{document}

% TU Eindhoven beamer template
% Author: Jens d'Hondt
% Eindhoven University of Technology

%%%%%%%%%%%%%%%%%%%%%%%%%%%%%%%%%%%%%%%%%%%%%%%%%%%%%%%%%%%%%%%%%%%%%%%%

\documentclass{beamer}
\usepackage[english]{babel}
\usepackage{calc}
\usepackage[absolute,overlay]{textpos}
\usepackage{graphicx}
\usepackage{subfig}
\usepackage{amsmath}
\usepackage{amsfonts}
\usepackage{amsthm}
\usepackage{mathtools}
\usepackage{comment}
\usepackage{MnSymbol,wasysym}

\newcommand{\defn}[1]{\textbf{\emph{#1}}}

%%%%%%%%%%%%%%%%%%%%%%%%%%%%%%%%%%%%%%%%%%%%%%%%%%%%%%%%%%%%%%%%%%%%%%%%

%% \setbeamertemplate%
%% {navigation symbols}{} % remove navigation symbols
%% \mode<presentation>{\usetheme{tue}}

%%%%%%%%%%%%%%%%%%%%%%%%%%%%%%%%%%%%%%%%%%%%%%%%%%%%%%%%%%%%%%%%%%%%%%%%

% BIB SETTINGS
\usepackage[backend=bibtex,firstinits=true,maxnames=30,maxcitenames=20,url=false,style=authoryear]{biblatex}
\bibliography{bibfile}
\setlength\bibitemsep{0.3cm} % space between entries in the reference list
\renewcommand{\bibfont}{\normalfont\scriptsize}
\setbeamerfont{footnote}{size=\tiny}
\renewcommand{\cite}[1]{\footnote<.->[frame]{\fullcite{#1}}}

%%%%%%%%%%%%%%%%%%%%%%%%%%%%%%%%%%%%%%%%%%%%%%%%%%%%%%%%%%%%%%%%%%%%%%%%

\title[]{CS112 at a Faster Pace}
%\title[]{The Title of the Presentation}
\institute[]{Computer Science Department \\ Boston University \\ Semester: Spring, 2025}
%\institute[]%
%{Eindhoven University of Technology, The Netherlands}
\author{Hongwei Xi}
%\author{First Author \and Second Author}
\date{\today}

%%%%%%%%%%%%%%%%%%%%%%%%%%%%%%%%%%%%%%%%%%%%%%%%%%%%%%%%%%%%%%%%%%%%%%%%

\begin{document}
{
\setbeamertemplate{footline}{\usebeamertemplate*{minimal footline}}
\frame{\titlepage}
}
{
\setbeamertemplate{footline}{\usebeamertemplate*{minimal footline}}
}
%%%%%%%%%%%%%%%%%%%%%%%%%%%%%%%%%%%%%%%%%%%%%%%%%%%%%%%%%%%%%%%%%%%%%%%%

\begin%
{frame}%
{What?}
\begin%
{itemize}%
\item
This faster-paced version of CS112 goes faster so that
more content can be covered. In particular, more focus will be on
teaching programming skills (besides data structures) in Java as well
as in general.
\item Textbook:\break
Algorithms by Robert Sedgewick and Kevin Wayne, Princeton University.
\item Here is an older version of CS112:\\[6pt]
  {\underline{\texttt{https://www.cs.bu.edu/~hwxi/hwxi-2020-12-31/}}}
  \break
  {\underline{\texttt{academic/courses/CS112/Spring10/classpage.html}}}
\end{itemize}
\end{frame}

%%%%%%%%%%%%%%%%%%%%%%%%%%%%%%%%%%%%%%%%%%%%%%%%%%%%%%%%%%%%%%%%%%%%%%%%

\begin%
{frame}%
{Why?}
\begin%
{itemize}
\item BUCS enrollments have gone up a lot!
\item There are many shortcomings to the {\em one-size-fits-all} model of teaching CS112.
\item A faster-paced version of CS112 can effectively address some of these shortcomings.
\end{itemize}
\end{frame}

%%%%%%%%%%%%%%%%%%%%%%%%%%%%%%%%%%%%%%%%%%%%%%%%%%%%%%%%%%%%%%%%%%%%%%%%

\begin%
{frame}%
{For whom?}

You may want to take it if
\begin%
{itemize}
\item
you enjoy programming in general, and/or
\item
you welcome more programming challenges, and/or
\item
you are already familiar with basic programming
\begin%
{itemize}
\item you are familiar with Java, and/or
\item you are familiar with another language like Python, and/or
\item you are confidant in your ability to pick up Java on your own.
\end{itemize}
\end{itemize}

\end{frame}

%%%%%%%%%%%%%%%%%%%%%%%%%%%%%%%%%%%%%%%%%%%%%%%%%%%%%%%%%%%%%%%%%%%%%%%%

\begin%
{frame}%
{What is it like?}

%% Unlike CS 112, which has plenty of TFs, CAs, office hours, and
%% discussion sections, Hongwei's section will have very little of
%% that. Also, as it's being taught for the first time, it will be more
%% subject to change; "experimental" may be the word to use.

\begin%
{itemize}
\item
This class is small. At most one TA is expected.
\item
As it is being taught for the first time, it will be more subject
to change. In short, it is somewhat "experimental".
\item
Grading tends to be interactive in the sense that you are given chances
to fix your programming errors. 
\end{itemize}

\end{frame}

%%%%%%%%%%%%%%%%%%%%%%%%%%%%%%%%%%%%%%%%%%%%%%%%%%%%%%%%%%%%%%%%%%%%%%%%

\begin%
{frame}
{How to proceed?}

\begin%
{itemize}
\item
There will be a placement test given.
\item
Please take it if you are interested. We will send out invitations
based on the quality of the submitted solutions to this test.
\item Note that the class time is not yet fixed. Here are some possibilities:
\begin%
{itemize}
\item Tuesday/Thursday 09:30am-10:45am
\item Tuesday/Thursday 11:00am-12:15am
\item Tuesday/Thursday 05:00pm-06:15pm
\end{itemize}

\end{itemize}

\end{frame}

%%%%%%%%%%%%%%%%%%%%%%%%%%%%%%%%%%%%%%%%%%%%%%%%%%%%%%%%%%%%%%%%%%%%%%%%

\end{document}

%%%%%%%%%%%%%%%%%%%%%%%%%%%%%%%%%%%%%%%%%%%%%%%%%%%%%%%%%%%%%%%%%%%%%%%%
%%%%%%%%%%%%%%%%%%%%%%%%%%%%%%%%%%%%%%%%%%%%%%%%%%%%%%%%%%%%%%%%%%%%%%%%

\documentclass[11pt]{article}
\usepackage{latexsym}
\oddsidemargin=-0.5cm
\evensidemargin=-0.5cm
\textwidth=17.60cm
\textheight=23.80cm
\topmargin=-1cm
\headsep=1cm

\title{BU~CAS~CS~392: Modern Compiler Construction in Python}
\author{(Syllabus)}
\date{}


\begin{document}
\maketitle
\thispagestyle{empty}

\begin{itemize}
\item {\bf Semester} Spring 2025
\item {\bf Instructor}: Hongwei Xi
\item {\bf Lecture Times}: TBD
\item {\bf Classroom}: TBD
\item {\bf Reference Book}:\kern6pt
{\em Modern Compiler Implementation in ML} by Andrew W. Appel.  ISBN
0-521-58274-1. Cambridge University Press.
\item {\bf Homepage}:~{\tt http://hwxi.github.io/TEACHING/CS392/2025S}
\item {\bf Overview}:
{\em Modern Compiler Construction in Python}
is a course that introduces students to some basics in the design and
implementation of compilers. In this course, we teach the theory
behind various components of a compiler as well as the programming techniques
involved to put the theory into practice. In particular, we adopt a style of
{\em modern} compiler construction that builds a compiler by stringing a sequence of
translations sharing a common closure-based interpreter-like structure.
The chosen programming language for implementation is Python~3.
However, you can seek the instructor's approval to choose a {\em functional}
programming language as your implementation language if you so wish.

\item {\bf Grades}
For instance, the final score may be calculated using the following formula.
\[\mbox{final score = 30\%$\cdot$(homework) + 30\%$\cdot$(midterm) + 30\%$\cdot$(final) + 10\%$\cdot$(participation)}\]
The final letter grade is calculated as follows.
\begin{itemize}
\item{\bf A}: final score is $85\%$ or above
\item{\bf B}: final score is $75\%$ or above
\item{\bf C}: final score is $65\%$ or above
\item{\bf D}: final score is $50\%$ or above
\end{itemize}

\item {\bf Academic Integrity}:
We adhere strictly to the standard BU guidelines for academic
integrity. For this course, it is perfectly acceptable for you to discuss
the general concepts and principles behind an assignment with other
students. However, it is not proper, without prior authorization of the
instructor, to arrive at collective solutions. In such a case, each student
is expected to develop, write up and hand in an individual solution and, in
doing so, gain a sufficient understanding of the problem so as to be able
to explain it adequately to the instructor.  Under {\em no} circumstances
should a student copy, partly or wholly, the completed solution of another
student. If one makes substantial use of certain code that is not written by
oneself, then the person must explicitly mention the source of the involved
code.

\end{itemize}

\end{document}

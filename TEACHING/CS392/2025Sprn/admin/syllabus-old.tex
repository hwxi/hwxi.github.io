\documentclass[11pt]{article}
\usepackage{latexsym}
\oddsidemargin=-0.5cm
\evensidemargin=-0.5cm
\textwidth=17.60cm
\textheight=23cm
\topmargin=-2cm
\headsep=1cm

\title{Introduction to CS II -- Data Structures\break CAS CS 112 Spring 2010}
\author{(Syllabus)}
\date{}

\begin{document}
\maketitle
\thispagestyle{empty}

\begin{itemize}
\item {\bf Semester} Spring 2010
\item {\bf Teaching Staff}
\begin{itemize}
\item
Instructor -- Hongwei Xi
\begin{itemize}
\item
Office Hours: TR: 5-6PM; W: 2-3PM
\item
Location: MCS 172, e-mail: \texttt{hwxi@cs.bu.edu}
\end{itemize}

\item
Teaching Fellow -- Konstantin Voevodski
\begin{itemize}
\item
Office Hours: M: 12:30-2:00PM; R: 2-3:30PM
\item
Location: PSY 223, e-mail: \texttt{kvodski@cs.bu.edu}
\end{itemize}
\end{itemize}

\item {\bf Lecture Times}: TR 12:30-2PM
\item {\bf Classroom}: GCB 209
\item {\bf Textbook}:\kern6pt
\begin{itemize}
\item
R. Sedgewick, Algorithms in Java (Parts 1-4), 3rd edition. Addison-Wesley, 2002, ISBN 0-201-36120-5.
\item
R. Sedgewick, Algorithms in Java (Part 5), 3rd edition. Addison-Wesley, 2003, ISBN 0-201-36121-3.
\end{itemize}

\item {\bf Homepage}: {\tt http://www.cs.bu.edu/\char126hwxi/academic/courses/CS112/Spring10}

\item {\bf Description}: This course starts by quickly revisiting, and then
building upon, advanced programming concepts in Java taught at the end of
CS~111. Then, the main focus of the course is on the design, analysis
and implementation of fundamental data structures used throughout computer
science. These include linked lists, stacks, queues, trees, hash tables,
graphs, as well as specialized methods for searching and sorting. All of
our implementations will be in the the object-oriented programming language
Java. The emphasis in teaching this course centers around the following:

\begin{itemize}
\item
Developing elegant and efficient code from an abstract specification;

\item
Literate programming (writing programs that can be read by humans as well
as machines);

\item
Developing a toolbox of advanced data structures for use in your future
programming tasks, and an awareness of various design patterns that recur
frequently in advanced programming;

\item
Critical thinking about programs and the programming process, which
involves:
\begin{itemize}
\item
Thinking about the best way to plan out the design using object-oriented
design and appropriate features of Java;
\item
Methodical and efficient development of the implementation using step-wise
refinement and incremental testing and debugging (using appropriate
debugging tools);
\item
Being able to convince yourself of the correctness of the implementation by
mathematical reasoning;
\item
Analyzing the running time (efficiency) of programs by inspection and
mathematical reasoning; and
\item
Evaluating the efficiency and correctness of programs empirically, by using
various tools in properly designed experiments.
\end{itemize}
\end{itemize}

\item {\bf Prerequisites}:

This course is designed for students who already program with a CS 111
level of proficiency in Java. If you do not have significant previous
exposure to programming, then you are requested to transfer to CS 111. You
are expected to be familiar with UNIX and EMACS (or other equivalent text
editor). Some help will be available in the section, but if you have not
used UNIX or EMACS before, then you should attend the appropriate tutorials
provided by B.U. Office of Information Technology.

\item {\bf Exams}:
\begin{itemize}
\item
First in-class midterm exam: Thursday, March 4, 2010 
\item
Second in-calss midterm exam: Thursday, April 15, 2010
\item
Final Exam: 9-11am, Friday, May 7, 2010 
\end{itemize}

\item {\bf Grades}
The final score is calculated using the following formula.
\[\mbox{final score = 30\%$\cdot$(homework) + 30\%$\cdot$(midterm) + 30\%$\cdot$(final) + 10\%$\cdot$(attendance + participation)}\]
The final letter grade is calculated as follows.
\begin{itemize}
\item{\bf A}: final score is $85\%$ or above
\item{\bf B}: final score is $75\%$ or above
\item{\bf C}: final score is $65\%$ or above
\item{\bf D}: final score is $50\%$ or above
\end{itemize}

\item
{\bf Program Submission}: Program assignments are to be submitted via the
{\em gsubmit} program. In case you need to learn how to use {\em gsubmit},
its documentation is available via a link on the homepage of this course.

\item
{\bf Attendance}: It is expected that you will attend the lectures and the
lab sessions for this course. I will take attendance at the beginning of
half of the lectures. I also ask that you arrive in class on time as it is
highly disruptive to have students flowing in throughout the class period.
Moreover, when a student is a bordline case, I will check the attendance
records before making a final determination.

\item
{\bf Academic Integrity}: We adhere strictly to the standard BU guidelines
for academic integrity. For this course, it is perfectly acceptable for you
to discuss the general concepts and principles behind an assignment with
other students. However, it is not proper, without prior authorization of
the instructor, to arrive at collective solutions. In such a case, each
student is expected to develop, write up and hand in an individual solution
and, in doing so, gain a sufficient understanding of the problem so as to
be able to explain it adequately to the instructor.  Under {\em no}
circumstances should a student copy, partly or wholly, the completed
solution of another student.

\end{itemize}

\end{document}

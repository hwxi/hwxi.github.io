\documentclass[11pt]{article}
\usepackage{latexsym}
\oddsidemargin=-0.5cm
\evensidemargin=-0.5cm
\textwidth=17.60cm
\textheight=23cm
\topmargin=-2cm
\headsep=1cm

\title{Introduction to CS II -- Data Structures\break CAS CS 392X1 Spring 2025}
\author{(Syllabus)}
\date{}

\begin{document}
\maketitle
\thispagestyle{empty}

\begin{itemize}
\item {\bf Semester} Spring 2025
\item {\bf Teaching Staff}
\begin{itemize}
\item
Instructor -- Hongwei Xi
\begin{itemize}
\item
Office Hours: TBA
\item
Location: CDS 727, e-mail: \texttt{hwxi@cs.bu.edu}
\end{itemize}

\item
Teaching Fellow -- TBA
\begin{itemize}
\item
Office Hours: TBA
\item
Location: TBA
\end{itemize}
\end{itemize}

\item {\bf Lecture Times}: TR 5-6:15PM
\item {\bf Classroom}: CDS 801
\item {\bf Textbook}:\kern6pt
\begin{itemize}
\item
Algorithms by R. Sedgewick and K. Wayne, 4th edition. Addison-Wesley, 2011.
\end{itemize}

\item {\bf Homepage}:
{\tt  https://hwxi.github.io/TEACHING/CS392/2025Sprn}

\item {\bf Description}:

This course starts by quickly revisiting and then building upon basic
programming concepts in Java. Then, the main focus of the course is on
the design, analysis and implementation of fundamental data structures
used throughout computer science. These include linked lists, stacks,
queues, trees, hash tables, graphs, as well as specialized methods for
searching and sorting. All of our implementations will be in the the
object-oriented programming language Java. The emphasis in teaching
this course centers around the following:

\begin{itemize}
\item
Developing elegant and efficient code from an abstract specification;

\item
Literate programming (writing programs that can be read by humans as well
as machines);

\item
Developing a toolbox of advanced data structures for use in your future
programming tasks, and an awareness of various design patterns that recur
frequently in advanced programming;

\item
Critical thinking about programs and the programming process, which
involves:
\begin{itemize}
\item
Thinking about the best way to plan out the design using object-oriented
design and appropriate features of Java;
\item
Methodical and efficient development of the implementation using step-wise
refinement and incremental testing and debugging (using appropriate
debugging tools);
\item
Being able to convince yourself of the correctness of the implementation by
mathematical reasoning;
\item
Analyzing the running time (efficiency) of programs by inspection and
mathematical reasoning; and
\item
Evaluating the efficiency and correctness of programs empirically, by using
various tools in properly designed experiments.
\end{itemize}
\end{itemize}

\item {\bf Prerequisites}:

This course is designed for students who already have a basic level of
proficiency in Java. If you do not have significant previous exposure
to programming, then you are requested to transfer to CS 111.  You are
expected to be familiar with a text editor (e.g., VIM and EMACS).

\item {\bf Exams}:
\begin{itemize}
\item
First midterm exam: TBA
\item
Second midterm exam: TBA
\item
The final exam date is yet to be anounced.
\end{itemize}

\item {\bf Grades}
The final score is calculated using the following formula.
\[\mbox{final score = 30\%$\cdot$(homework) + 30\%$\cdot$(midterm) + 35\%$\cdot$(final) + 5\%$\cdot$(participation)}\]
The final letter grade is calculated as follows.
\begin{itemize}
\item{\bf A}: final score is $85\%$ or above
\item{\bf B}: final score is $75\%$ or above
\item{\bf C}: final score is $65\%$ or above
\item{\bf D}: final score is $50\%$ or above
\end{itemize}

\item
{\bf Program Submission}: Program assignments are to be submitted via the
{\em gsubmit} program. In case you need to learn how to use {\em gsubmit},
its documentation is available via a link on the homepage of this course.

\item
{\bf Attendance}:
It is expected that you will attend the lectures and the lab sessions
for this course.

\item
{\bf Academic Integrity}:
We adhere strictly to the standard BU guidelines for academic
integrity. For this course, it is perfectly acceptable for you to discuss
the general concepts and principles behind an assignment with other
students. However, it is not proper, without prior authorization of the
instructor, to arrive at collective solutions. In such a case, each student
is expected to develop, write up and hand in an individual solution and, in
doing so, gain a sufficient understanding of the problem so as to be able
to explain it adequately to the instructor.  Under {\em no} circumstances
should a student copy, partly or wholly, the completed solution of another
student. If one makes substantial use of certain code that is not written by
oneself, then the person must explicitly mention the source of the involved
code.

\end{itemize}

\end{document}

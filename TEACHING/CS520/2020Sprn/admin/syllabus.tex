\documentclass[11pt]{article}
\usepackage{latexsym}
\oddsidemargin=-0.5cm
\evensidemargin=-0.5cm
\textwidth=17.60cm
\textheight=23.80cm
\topmargin=-1cm
\headsep=1cm

\title{BU CAS CS 520: Principles of Programming Languages}
\author{(Syllabus)}
\date{}


\begin{document}
\maketitle
\thispagestyle{empty}

\begin{itemize}
\item {\bf Semester} Spring 2019
\item {\bf Instructor}: Hongwei Xi
\item {\bf Lecture Times}: Tue/Thu 2:00-3:15
\item {\bf Office Hours}: Mon: 3:00-5:00; Tue: 3:30-4:30; or by appointment
\item {\bf Classroom}: MCS B25

\item {\bf Reference Books}:\kern6pt
\begin{itemize}
\item
{\em Introduction to Programming in ATS} by Hongwei Xi
\item
{\em Crash into Functional Programming via ATS} by Hongwei Xi
\item
{\em Practical Foundations for Programming Languages (draft)} by Robert Harper
\item
{\em Types and Programming Languages} by Benjamin C. Pierce.  ISBN 0-262-16209-1. MIT Press.
\end{itemize}

\item {\bf Homepage}: {\tt http://www.cs.bu.edu/\char126hwxi/academic/courses/Spring19/CS520.html}

\item {\bf Overview}:

{\em Principles of Programming Languages} is a course
that introduces students to some fundamental principles and techniques in
modern programming language design and implementation. The course mainly
covers type theories in programming and emphasizes the need for
mathematical formalism in both describing and analyzing programming
languages and programs.

\item {\bf Class Schedule}
Please find on the class homepage a detailed class schedule by each week.

\item {\bf Grades}
The final score is calculated using the following formula.
\[\begin{array}{l}
\mbox{final score = 20\%$\cdot$(HW) + 20\%$\cdot$(MT1) + 20\%$\cdot$(MT2) + 30\%$\cdot$(final) + 10\%$\cdot$(participation)}
\end{array}\]
The final letter grade is calculated as follows.
\begin{itemize}
\item{\bf A}: final score is $80\%$ or above (A, A-)
\item{\bf B}: final score is $70\%$ or above (B+, B, B-)
\item{\bf C}: final score is $60\%$ or above (C+, C, C-)
\item{\bf D}: final score is $50\%$ or above (D)
\end{itemize}

% \item {\bf Academic Integrity}: We adhere strictly to the standard BU
% guidelines for academic integrity. For this course, it is perfectly
% acceptable for you to discuss the general concepts and principles behind an
% assignment with other students. However, it is not proper, without prior
% authorization of the instructor, to arrive at collective solutions. In such
% a case, each student is expected to develop, write up and hand in an
% individual solution and, in doing so, gain a sufficient understanding of
% the problem so as to be able to explain it adequately to the instructor.
% Under {\em no} circumstances should a student copy, partly or wholely, the
% completed solution of another student.

\item{\bf Academic Integrity}:
We adhere strictly to the standard BU guidelines for academic
integrity. For this course, it is perfectly acceptable for you to discuss
the general concepts and principles behind an assignment with other
students. However, it is not proper, without prior authorization of the
instructor, to arrive at collective solutions. In such a case, each student
is expected to develop, write up and hand in an individual solution and, in
doing so, gain a sufficient understanding of the problem so as to be able
to explain it adequately to the instructor.  Under {\em no} circumstances
should a student copy, partly or wholly, the completed solution of another
student. If one makes substantial use of certain code that is not written by
oneself, then the person must explicitly mention the source of the involved
code.

\end{itemize}

\end{document}
